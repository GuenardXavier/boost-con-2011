\section{SIMD and STL}
\subsection{Rationale}
\frame
{
  \frametitle{Operations vs Data}
  \begin{block}{Where/How to store our data ?}
  \begin{itemize}
  \item SIMD operations require data to operate onto
  \item Usual approach force a specific container type onto users
  \item This is bad because it is overspecified
  \end{itemize}
  \end{block}{}

  \begin{block}{A better approach}<2->
  \begin{itemize}
  \item SIMD compliant allocators
  \item SIMD Range and Iterators over ContiguousRange
  \item Adapt our SIMD classes to work with a subset of STD algorithms
  \end{itemize}
  \end{block}{}
}


\subsection{SIMD Allocator}
\frame
{
  \frametitle{SIMD allocators}
  \begin{block}{Rationale}
  \begin{itemize}
  \item Allow containers to handle memory in a SIMD compliant way
  \item Handles alignement of memory
  \item Handles padding of memory
  \end{itemize}
  \end{block}{}

  \begin{block}{Example}
  \begin{center}
  \lstallocator
  \end{center}
  \end{block}{}
}

\subsection{SIMD Iterator and Ranges}
\frame
{
  \frametitle{From Range to SIMDRange}
  \begin{block}{Iterator interface}
  \begin{itemize}
  \item Boost.SIMD provides \texttt{simd::begin()}/\texttt{simd::end()}
  \item Turn iterators in SIMD iterators returning \texttt{pack}
  \item Take a regular range, iterate over it in SIMD
  \end{itemize}
  \end{block}

  \begin{block}{Example}<2->
  \begin{center}
  \only<2>{\lststlrange}
  \only<3>{\lststlrangeb}
  \end{center}
  \end{block}
}

\frame
{
  \frametitle{From Range to SIMDRange}
  \begin{block}{Iterator interface}
  \begin{itemize}
  \item \texttt{native} and \texttt{pack} provides \texttt{begin()}/\texttt{end()}
  \item Directly usable in STD algorithms
  \item Directly usable in Boost.Range algorithms
  \end{itemize}
  \end{block}{}

  \begin{block}{Example}
  \begin{center}
  \lstpackrange
  \end{center}
  \end{block}{}
}

\frame
{
  \frametitle{SIMD values as Range}
  \begin{block}{Putting everythign together}
  \begin{center}
  \lstranges
  \end{center}
  \end{block}{}
}

\subsection{SIMD Algorithms}

\frame
{

}

\frame
{

}
