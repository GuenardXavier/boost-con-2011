
\section{Interface}
\subsection{Hand-written SIMD}
\begin{frame}[fragile]
	\frametitle{Writing it by hand}

	\begin{block}{Doing \texttt{a * b + c} with vectors of 32-bit integers : SSE}
	\lstsse
	\end{block}{}
	
	\begin{block}{Doing \texttt{a * b + c} with vectors of 32-bit integers : Altivec}
	\lstvmx
	\end{block}{}
\end{frame}

\subsection{Pack}
\frame
{
	\frametitle{The pack abstraction}
	\begin{block}{\texttt{simd::pack<T>}}
	\begin{tabular}{ll}
	\texttt{pack<T, N>} & SIMD register that packs \texttt{N} elements of type \texttt{T}\\
	\texttt{pack<T>} & automatically finds best \texttt{N} available
	\end{tabular}
	\begin{itemize}
\footnotesize
	\item Behaves just like \texttt{T} except operations yield a pack of \texttt{T} and not a \texttt{T}.
\end{itemize}
	\end{block}{}
	
	\begin{block}{Constraints}
         \begin{itemize}
\footnotesize
\item	\texttt{T} must be a fundamental arithmetic type, i.e. (\lstinline{un})\lstinline{signed char}, (\lstinline{unsigned}) \lstinline{short}, (\lstinline{unsigned}) \lstinline{int}, (\lstinline{unsigned}) \lstinline{long}, (\lstinline{unsigned}) \lstinline{long long}, \lstinline{float} or \lstinline{double} -- not \lstinline{bool}.
	
	\item \texttt{N} must be a power of 2.
	\end{itemize}
       \end{block}{}
}


\subsection{Primitives}

\frame
{
  \frametitle{\texttt{pack} API}
	
  \begin{block}{Operators}
	\begin{itemize}
   \footnotesize
		\item All overloadable operators are available
		\item \lstinline{pack<T>} x \lstinline{pack<T>} operations but also \lstinline{pack<T>} x \lstinline{T}
		\item Type coercion and promotion disabled\\
		      \lstinline{uint8_t(255) + uint8_t(1)} yields \lstinline{uint8_t(0)}, not \lstinline{int(256)}
	\end{itemize}	
	\end{block}{}

	\begin{block}{Comparisons}
	 \begin{itemize}
\footnotesize
	\item \texttt{==, !=, <, <=,>} and \texttt{>=}) perform lexical comparisons.
	\item \texttt{eq,neq,lt,gt,le} and \texttt{ge} as functions return \texttt{pack} of boolean.
\end{itemize}
\end{block}{}
\begin{block}{Other properties}
	\begin{itemize}
   \footnotesize
		\item Models both a \texttt{ReadOnlyRandomAccessFusionSequence} and \texttt{ReadOnlyRandomAccessRange}
		\item \lstinline{at\_c<i>(p)} or \lstinline{p[i]} can be used to access the i-th element, but is usually slow (\lstinline{at\_c} is faster)
	\end{itemize}
	\end{block}{}
}

\frame
{
  \frametitle{\texttt{pack} API}
	
	\begin{block}{Memory access}
\footnotesize
 Memory must be aligned on \lstinline{sizeof(T)*N} to load/store a \lstinline{pack<T, N>}
	from or to a \lstinline{T*}. Errors leads to undefined behaviors.
\end{block}{}

\begin{block}{Examples}<2->
\only<2>
{
\footnotesize
\lstinline{load< pack<T, N> >(p, i)} loads pack at aligned address \lstinline{p + i*N}
\begin{center}\pgfuseimage{load}\end{center}
}

\only<3>
{
\footnotesize
\lstinline{load< pack<T, N>, Offset>(p, i)} loads pack at address \lstinline{p + i*N + Offset}, \lstinline{p + i} must be aligned.
\begin{center}\pgfuseimage{load-offset}\end{center}
}

\only<4>
{
\footnotesize
  \lstinline{store(p, i, pk)} stores pack \lstinline{pk} at aligned address \lstinline{p + i*N}
}
\end{block}{}
}

\frame
{
  \frametitle{\texttt{pack} as a \texttt{proto} entity}
	
	\begin{block}{Rationale}
	\begin{itemize}\footnotesize
       \item Most SIMD ISA have fused operations (FMA, etc...)
     \item We want ot write simple code but yet get best performances out of these
  \item We need lazy evaluation : \texttt{proto} to the rescue
        \end{itemize}
	\end{block}{}
	
	\begin{block}{Advantage}
	\begin{itemize}\footnotesize
		\item All expressions, even those involving functions,
		      generate template expressions that are evaluated on assignment
		      or in the conversion operator
		\item \lstinline{a * b + c} is mapped to \lstinline{fma(a, b, c)}\\
		      \lstinline{a + b * c} is mapped to \lstinline{fma(b, c, a)}\\
		      \lstinline{!(a < b)} is mapped to \lstinline{is_nle(a, b)}
   \item the optimisation system is open for extensions
	\end{itemize}
	\end{block}{}
}

\begin{frame}
	\frametitle{Extra arithmetic, bitwise and ieee operations, predicates}
	
	\begin{multicols}{3}		
	
		\textbf{Arithmetic}
		\begin{itemize}
			\item saturated arithmetic
			\item float/int conversion
			\item round, floor, ceil, trunc
			\item sqrt, hypot
			\item average
			\item random
			\item min/max
			\item rounded division and remainder
		\end{itemize}

		\textbf{Bitwise}
		\begin{itemize}			
			\item select
			\item andnot, ornot
			\item popcnt
			\item ffs
			\item ror, rol
			\item rshr, rshl
			\item twopower
		\end{itemize}
			\vfill
		\textbf{IEEE}
		\begin{itemize}
			\item ilogb, frexp
			\item ldexp
			\item next/prev
			\item ulpdist
		\end{itemize}
			
		\textbf{Predicates}
		\begin{itemize}
			\item comparison with zero
			\item negation of comparison
			\item is\_unord, is\_nan, is\_invalid
			\item is\_odd, is\_even
			\item majority
		\end{itemize}
		
		\vfill
	
	\end{multicols}
	
\end{frame}

\begin{frame}
	\frametitle{Reduction and SWAR operations}
	
	\begin{multicols}{2}	
	
	\textbf{Reduction}
	\begin{itemize}
		\item any, all
		\item nbtrue
		\item minimum/maximum, posmin/posmax
		\item sum
		\item product, dot product
	\end{itemize}
	\columnbreak	
	
	\textbf{SWAR}
	\begin{itemize}
		\item group/split
		\item splatted reduction
		\item cumsum
		\item sort
	\end{itemize}
	
	\vfill
	\end{multicols}
	
\end{frame}

\subsection{Extension Points}
\frame
{
	\frametitle{\texttt{native<T, X>} : SIMD register of \texttt{T} on arch. \texttt{X}}
	
      	\begin{block}{Semantic}
	\begin{itemize}\footnotesize
		\item like \lstinline{pack} but Plain Old Data and all operations and functions return values and not
		      expression templates.
		\item \lstinline{X} characterizes the register type, not the instructions available. Only one tag for all
		      SSE variants.
		\item It is the interface that must be used to extend the library.
	\end{itemize}
	\end{block}{}
	
        \begin{block}{Example}\footnotesize
	\begin{tabular}{ll}
	\lstinline{native<float, tag::sse_>} & wraps a \lstinline{__m128}\\
	\lstinline{native<uint8_t, tag::sse_>} & wraps a \lstinline{__m128i}\\
	\lstinline{native<double, tag::avx_>} & wraps a \lstinline{__m256d}\\
	\lstinline{native<float, tag::altivec_>} & wraps a \lstinline{__vector float}
	\end{tabular}
	\end{block}{}
}

\frame
{
\frametitle{\texttt{native<T, X>} : SIMD register of \texttt{T} on arch. \texttt{X}}
		
\begin{block}{Software fallback}
	\begin{itemize}\footnotesize
		\item \lstinline{tag::none_<N>} is a software-emulated SIMD architecture with a register size of \lstinline{N} bytes
		\item It is used as fallback when no satisfying SIMD architecture is found
		\item Thanks to this, code can degrade well and remain portable.
	\item Default native type when no SIMD is found : \lstinline{native<T, tag::none_<8> >}
	
	\end{itemize}
	\end{block}{}	
	}

\subsection{Set-up}
\frame
{
	\frametitle{How to compile the examples?}	
	
	Pre-requisites:
	\begin{itemize}
		\item Python 2.6+
		\item CMake 2.6+
		\item Git 1.6+
		\item Boost 1.46+ or SVN trunk
		\item $NT^2$ git master
		\item Preferably a Linux/x86/GCC setup with GCC 4.6+
	\end{itemize}
	\bigskip
	
	\texttt{BOOST\_ROOT} must point to Boost\\
	\texttt{NT2\_SOURCE\_ROOT} must point to $NT^2$
	\bigskip
	
	Examples are at:\\
	\texttt{git://github.com/MetaScale/boost-con-2011.git}
}

\subsection{RGB to grayscale}
\begin{frame}[fragile]
	\frametitle{RGB to grayscale}
	
	Data:\\
\bigskip
	\begin{lstlisting}
	float const *red, *green, *blue;
	float* result;
	\end{lstlisting}
	
\bigskip
	Scalar version:\\
\bigskip
	\begin{lstlisting}
	for(std::size_t i = 0; i != height*width; ++i)
	    result[i] = 0.3f * red[i] + 0.59f * green[i] + 0.11f * blue[i]; 
	\end{lstlisting}
	
\end{frame}

\begin{frame}[fragile]
	\frametitle{SIMD version}
	
	\begin{lstlisting}
	std::size_t N = meta::cardinal_of<pack<float>>::value;
	for(std::size_t i = 0; i != height*width/N; ++i)
	{
	    pack<float> r  = load< pack<float> >(red, i);
	    pack<float> g = load< pack<float> >(green, i);
	    pack<float> b = load< pack<float> >(blue, i);
		
	    pack<float> res = 0.3f * r + 0.59f * g + 0.11f * b;
	    store(res, result, i);
	}
	\end{lstlisting}
	
\end{frame}

\begin{frame}
	\frametitle{Easy enough, but what if...}

	\begin{itemize}
		\item ... i've got interleaved RGB or RGBA?
		\item ... i've got 8-bit integers and not floats?
	\end{itemize}
	
	Can be complicated, we'll see that later.
	
\end{frame}