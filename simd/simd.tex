\documentclass{beamer}

\mode<presentation>
{
  \usetheme{Darmstadt}
  % or ...

  \setbeamercovered{transparent}
  % or whatever (possibly just delete it)
}
\setbeamertemplate{navigation symbols}{} 

\usepackage[utf8x]{inputenc}
\usepackage{amsmath} 
\usepackage{listings}
\usepackage{multicol}

%\usepackage{times}
%\usepackage[T1]{fontenc}
% Or whatever. Note that the encoding and the font should match. If T1
% does not look nice, try deleting the line with the fontenc.

\lstset{basicstyle=\ttfamily}


\title%[Short Paper Title] % (optional, use only with long paper titles)
{Practical SIMD acceleration with Boost.SIMD}

%\subtitle
%{Include Only If Paper Has a Subtitle}

\author%[Author, Another] % (optional, use only with lots of authors)
{Mathias Gaunard  \and Joël Falcou}
% - Give the names in the same order as the appear in the paper.
% - Use the \inst{?} command only if the authors have different
%   affiliation.

\institute[MetaScale] % (optional, but mostly needed)
{MetaScale Inc.}
% - Use the \inst command only if there are several affiliations.
% - Keep it simple, no one is interested in your street address.

\date%[CFP 2003] % (optional, should be abbreviation of conference name)
{Boostcon 2011}
% - Either use conference name or its abbreviation.
% - Not really informative to the audience, more for people (including
%   yourself) who are reading the slides online

\subject{Computer Science}
% This is only inserted into the PDF information catalog. Can be left
% out. 



% If you have a file called "university-logo-filename.xxx", where xxx
% is a graphic format that can be processed by latex or pdflatex,
% resp., then you can add a logo as follows:

\pgfdeclareimage[height=0.5cm]{university-logo}{logo}
\logo{\pgfuseimage{university-logo}}



% Delete this, if you do not want the table of contents to pop up at
% the beginning of each subsection:
%\AtBeginSubsection[]
%{
%  \begin{frame}<beamer>
%    \frametitle{Outline}
%    \tableofcontents[currentsection,currentsubsection]
%  \end{frame}
%}


% If you wish to uncover everything in a step-wise fashion, uncomment
% the following command: 

%\beamerdefaultoverlayspecification{<+->}


\begin{document}

\begin{frame}
\titlepage
\end{frame}

\section{Introduction}

\subsection{SIMD}
\begin{frame}
	\frametitle{What's SIMD?}

	\begin{itemize}
		\item Single Instruction, Multiple Data
		\item Operations you can perform on NxT elements
		      packed within a single register
		\item Up to N times faster than doing it element per element
		      with the ALU or the FPU
	\end{itemize}

\end{frame}

\subsection{SIMD abstraction}
\begin{frame}
	\frametitle{Why is SIMD abstraction needed?}
	
	\begin{multicols}{2}
	
		x86 family
		\begin{itemize}
			\item MMX 64-bit float, double
			\item SSE 128-bit float
			\item SSE2 128-bit int8, int16, int32, int64, double
			\item SSE3
			\item SSSE3
			\item SSE4a (AMD only)
			\item SSE4.1
			\item SSE4.2
			\item AVX 256-bit float, double
			\item FMA4 (AMD only)
			\item XOP (AMD only)
			\item FMA3
		\end{itemize}
		\columnbreak
		
		PowerPC/VMX family
		\begin{itemize}
			\item AltiVec 128-bit int8, int16, int32, int64, float
			\item Cell SPU 128-bit int8, int16, int32, int64, float, double
		\end{itemize}
		
		ARM family
		\begin{itemize}
			\item VFP 64-bit float, double
			\item NEON 64-bit and 128-bit float, int8, int16, int32, int64
		\end{itemize}
		
	\vfill
	\end{multicols}
		
\end{frame}

\subsection{Explicit SIMD parallelization}
\begin{frame}
	\frametitle{Why not let the compiler do it?}
	
	\begin{itemize}
		\item Compilers are only so smart
		\item Automatic vectorization can only happen if:
		\begin{itemize}
			\item Memory is well agenced
			\item Code is inherently vectorizable
		\end{itemize}
	\end{itemize}
	
	\begin{itemize}
		\item Compilers don't always have enough static information to know what they can vectorize
		\item Designing for vectorization is a human process
	\end{itemize}
	
	Conclusion:
	\begin{itemize}
		\item Declaring SIMD parallelism explicitly is the best way
              to ensure your code gets vectorized
        \item To be demonstrated by this presentation
    \end{itemize}

\end{frame}

\subsection{Getting Boost.SIMD}
\begin{frame}
	\frametitle{Where to get Boost.SIMD?}
	
	\begin{itemize}
		\item No Boost.SIMD library yet
		\item Spin-off of $NT^2$
		\item Project of turning it into a Boost library ongoing\\
		      GSoC project starting in July
    \end{itemize}
	\bigskip	
	
	Get $NT^2$ at \texttt{git://github.com/MetaScale/nt2.git}
	
\end{frame}

\subsection{Set-up}
\begin{frame}
	\frametitle{How to compile the examples?}	
	
	Pre-requisites:
	\begin{itemize}
		\item Python 2.6+
		\item CMake 2.6+
		\item Git 1.6+
		\item Boost 1.46+ or SVN trunk
		\item $NT^2$ git master
		\item Preferably a Linux/x86/GCC setup with GCC 4.6+
	\end{itemize}
	\bigskip
	
	\texttt{BOOST\_ROOT} must point to Boost\\
	\texttt{NT2\_SOURCE\_ROOT} must point to $NT^2$
	\bigskip
	
	Examples are at:\\
	\texttt{git://github.com/MetaScale/boost-con-2011.git}
	
\end{frame}

%\begin{frame}
%  \frametitle{Outline}
%\tableofcontents
  % You might wish to add the option [pausesections]
%\end{frame}

\section{Interface}
\subsection{Pack}

\begin{frame}
	\frametitle{Pack}
	
	\begin{tabular}{ll}
	\lstinline{pack<T, N>} & SIMD register that packs \lstinline{N} elements of type \lstinline{T}\\
	\lstinline{pack<T>} & automatically finds best \lstinline{N} available
	\end{tabular}
	\bigskip	
	
	\begin{quote} Behaves just like \lstinline{T} except operations yield a pack of \lstinline{T} and not a \lstinline{T}. \end{quote}
	\bigskip	
	
	\lstinline{T} must be a fundamental arithmetic type, i.e. (\lstinline{un})\lstinline{signed char}, (\lstinline{unsigned}) \lstinline{short}, (\lstinline{unsigned}) \lstinline{int}, (\lstinline{unsigned}) \lstinline{long}, (\lstinline{unsigned}) \lstinline{long long}, \lstinline{float} or \lstinline{double} -- not \lstinline{bool}.
	\bigskip
	
	\lstinline{N} must be a power of 2.
	
\end{frame}

\begin{frame}
	\frametitle{Element access}
	
	\begin{itemize}
		\item like \lstinline{boost::array}, \lstinline{pack} is a random access read-only fusion sequence and range
		\item \lstinline{at\_c<i>(p)} or \lstinline{p[i]} can be used to access the i-th element, but is usually slow (\lstinline{at\_c} faster, can avoid going through memory)
	\end{itemize}
	
\end{frame}

\begin{frame}
	\frametitle{Proto expressions}
	
	\begin{itemize}
		\item All expressions, even those involving functions,
		      generate template expressions that are evaluated on assignment
		      or in the conversion operator
		\item \lstinline{a * b + c} is mapped to \lstinline{fma(a, b, c)}\\
		      \lstinline{a + b * c} is mapped to \lstinline{fma(b, c, a)}\\
		      \lstinline{!(a < b)} is mapped to \lstinline{is_nle(a, b)}
	\end{itemize}
	
\end{frame}

\subsection{Native}
\begin{frame}
	\frametitle{Under the hood}
	
	\begin{tabular}{ll}
	\lstinline{native<T, X>} & SIMD register for type \lstinline{T} on architecture \lstinline{X}
	\end{tabular}
	\bigskip
	
	\begin{itemize}
		\item like \lstinline{pack} but Plain Old Data and all operations and functions return values and not
		      expression templates.
		\item \lstinline{X} characterizes the register type, not the instructions available. Only one tag for all
		      SSE variants.
		\item It is the interface that must be used to extend the library.
	\end{itemize}
	\bigskip
	
	\begin{tabular}{ll}
	\lstinline{native<float, tag::sse_>} & wraps a \lstinline{__m128}\\
	\lstinline{native<uint8_t, tag::sse_>} & wraps a \lstinline{__m128i}\\
	\lstinline{native<double, tag::avx_>} & wraps a \lstinline{__m256d}\\
	\lstinline{native<float, tag::altivec_>} & wraps a \lstinline{__vector float}
	\end{tabular}
	
\end{frame}

\begin{frame}
	\frametitle{Software fallback}
	
	\begin{itemize}
		\item \lstinline{tag::none_<N>} is a software-emulated SIMD architecture with a register size of \lstinline{N} bytes
		\item It is used as fallback when no satisfying SIMD architecture is found
		\item Thanks to this, code can degrade well and remain portable.
	\end{itemize}
	\bigskip	
	
	Default native type when no SIMD is found is\\
	\hspace{10px}\lstinline{native<T, tag::none_<8> >}
	
\end{frame}

\subsection{Primitives}

\begin{frame}
	\frametitle{Operators}
	
	\begin{itemize}
		\item All overloadable operators are available
		\item \lstinline{pack<T>} x \lstinline{pack<T>} operations but also \lstinline{pack<T>} x \lstinline{T}
		\item All operators also exist as functions and PFOs
		\item \lstinline{if_else} also available, but evaluation not lazy -- all branches get evaluated
		\item Type coercion and promotion disabled\\
		      \lstinline{uint8_t(255) + uint8_t(1)} yields \lstinline{uint8_t(0)}, not \lstinline{int(256)}
		\item Comparison operators are lexicographic so as to yield booleans,
		      use other functions if you want to do a memberwise comparison.
	\end{itemize}	
	
\end{frame}

\begin{frame}
	\frametitle{Memory access}
	
	Memory must be aligned on \lstinline{sizeof(T)*N} to load/store a \lstinline{pack<T, N>}
	from a \lstinline{T*}\\
	-- except when it is statically known by how many elements it is not aligned.
	
	\begin{itemize}
		\item \lstinline{load< pack<T, N> >(p, i)} loads pack at aligned address \lstinline{p + i*N}
		\item \lstinline{load< pack<T, N>, Offset>(p, i)} loads pack at address \lstinline{p + i*N}, \lstinline{p + Offset} must be aligned
		\item \lstinline{store(p, i, pk)} stores pack \lstinline{pk} at aligned address \lstinline{p + i*N}
	\end{itemize}
\end{frame}

\begin{frame}
	\frametitle{Extra arithmetic, bitwise and ieee operations, predicates}
	
	\begin{multicols}{3}		
	
		Arithmetic
		\begin{itemize}
			\item saturated arithmetic
			\item float/int conversion
			\item round, floor, ceil, trunc
			\item sqrt, hypot
			\item average
			\item random
			\item min/max
			\item rounded division and remainder
		\end{itemize}

		Bitwise
		\begin{itemize}			
			\item select
			\item andnot, ornot
			\item popcnt
			\item ffs
			\item ror, rol, rshr, rshl
			\item twopower
		\end{itemize}
			
		IEEE
		\begin{itemize}
			\item ilogb, frexp
			\item ldexp
			\item next/prev
			\item ulpdist
		\end{itemize}
			
		Predicates
		\begin{itemize}
			\item comparison with zero
			\item negation of comparison
			\item is\_unord, is\_nan, is\_invalid
			\item is\_odd, is\_even
			\item majority
		\end{itemize}
		
		\vfill
	
	\end{multicols}
	
\end{frame}

\begin{frame}
	\frametitle{Reduction and SWAR operations}
	
	\begin{multicols}{2}	
	
	Reduction
	\begin{itemize}
		\item any, all
		\item nbtrue
		\item hmsb
		\item minimum/maximum, posmin/posmax
		\item sum
		\item product, dot product
	\end{itemize}
	\columnbreak	
	
	SWAR
	\begin{itemize}
		\item group/split
		\item splatted reduction
		\item cumsum
		\item sort
	\end{itemize}
	
	\vfill
	\end{multicols}
	
\end{frame}

\section{Fast memcpy}

\subsection{Naive memcpy}

\begin{frame}
\end{frame}

\end{document}
