\frame
{
  \frametitle{BSML primitives}
  \begin{block}{Distributed Vector}
  \footnotesize
   A BSP distributed vector is a vector where each element lives on a different BSP node
  \begin{itemize}
  \footnotesize
  \item \texttt{<<v>>} : build a vector from value or a function $v$
  \item \texttt{\$v\$} : access to the local vector element
  \item A parallel vector of type $'a$ has type $par 'a$
  \end{itemize}
  \end{block}
}

\frame
{
  \frametitle{BSML primitives}
  \begin{block}{The \texttt{proj} function}
  \footnotesize
   \begin{itemize}
  \footnotesize
  \item Replicates a parallel vector around all BSP nodes
  \item Prototype: $proj : par\ 'a -> par\ (int-> 'a)$
  \end{itemize}
  \end{block}

  \begin{block}{Semantic of $proj \, v$}
  \begin{center}\pgfuseimage{bsp_proj}\end{center}
  \end{block}
}

\frame
{
  \frametitle{BSML primitives}
  \begin{block}{The \texttt{put} function}
  \footnotesize
   \begin{itemize}
  \footnotesize
  \item Generic all-to-all communications function
  \item Prototype: $put : par\ (int\ ->\ 'a)\ ->\ par\ (int\ ->\ 'a)$
  \end{itemize}
  \end{block}

  \begin{block}{Semantic of $put \, vf$}
  \begin{center}\pgfuseimage{bsp_put}\end{center}
  \end{block}
}

\frame
{
  \frametitle{A sample BSML Code}
  \begin{block}{BSML Inner Product}
  \lstbsml
  \end{block}

  \begin{block}{How does it works}<2->
  \begin{itemize}
  \footnotesize
  \item Build a distributed vector from $v[i]^2$ in parallel
  \item Exchange partial results with all nodes
  \item Perform a final reduction
  \end{itemize}
  \end{block}
}

\frame
{
  \frametitle{From BSML to BSP++}
  \begin{block}{Why looking at BSML}  
  \begin{itemize}
  \footnotesize
   \item Provides a compact and abstract interface
   \item BSML likes playing with lambda and so do we
  \end{itemize}
  \end{block}{}

  \begin{block}{The Plan}  
  \begin{itemize}
  \footnotesize
   \item Implement BSML interface and abstraction ic C++
   \item Try to work on the functionnal side to limit errors
   \item Try to play nice with C++ functionnal idioms
  \end{itemize}
  \end{block}{}
}


\frame
{
  \frametitle{BSP++ 101}
  \begin{block}{Main Program Structure}
  \begin{itemize}
  \footnotesize
  \item Managed main handles parallel runtime
  \item Everything in a BSP programm is parallel
  \end{itemize}
  \end{block}{}

  \begin{block}{Example}<2->
  \lstbspmain
  \end{block}{}
}

\frame
{
  \frametitle{BSP++ primitives}
  \begin{block}{Parallel vector : \texttt{par<T>}}
  \begin{itemize}
  \footnotesize
  \item \texttt{par<T>} is a BSP distributed \texttt{T}
  \item Constructible from values, functions and ranges
  \end{itemize}
  \end{block}{}

  \begin{block}{\texttt{par<T>} Interface}<2->
  \only<2>{\lstparctor}
  \only<3>{\lstparaccess}
  \end{block}{}
}

\frame
{
  \frametitle{BSP++ primitives}
  \begin{block}{The \texttt{proj} and \texttt{put} function}
  \begin{itemize}
  \footnotesize
  \item BSML returns function \textbf{value}
  \item Let's return \textbf{Callable Object} embedding the result
  \item Make them Range for easier interoperability 
  \end{itemize} 
  \end{block}{}

  \begin{block}{Examples}<2->
  \only<2>{\lstbspproj}
  \only<3>{\lstbspput}
  \end{block}{}
}

\frame
{
  \frametitle{A sample BSP++ code}
  \begin{block}{BSP++ Inner Product}
   \lstbspinnerproduct
  \end{block}{}
}

\frame
{
  \frametitle{Support for Hybrid programming}
  \begin{center}
  \pgfuseimage{hybrid}
  \end{center}
}

\frame
{
  \frametitle{Support for Hybrid programming}
  \begin{block}{BSP++ Hybrid Inner Product}
  %\lsthybridinnerprod
  \end{block}{}
}



