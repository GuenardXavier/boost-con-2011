\frame
{
  \frametitle{Programming Tools and Models}
  \begin{block}{Message Passing Interface (MPI)}
  \begin{itemize}
  \footnotesize
  \item Run multiple process across distributed nodes
  \item Process use \textbf{Message} to communicate
  \item Provides a set of ready-to-use communications primitives
 \end{itemize}
  \end{block}{}

  \begin{block}{OpenMP}
  \begin{itemize}
  \footnotesize
  \item Standard language extension for shared memory system
  \item Parallelism is expressed as \textbf{parallel sections} using \texttt{\#pragma}
  \item Provides functions for threads handling and synchronization
 \end{itemize}
  \end{block}{}
}

\frame
{
  \frametitle{Higher Level Models}
  \begin{block}{What do we need}
  \begin{itemize}
  \footnotesize
  \item Architecture asbtraction
  \item Performances estimation
  \item Easy to use for the end user
 \end{itemize}
  \end{block}{}

  \begin{block}{What's available ?}<2->
  \begin{itemize}
  \footnotesize
  \item Stream processing
  \item Parallel Skeletons
  \item \alert<3->{Bulk Synchronous Parallelism}
 \end{itemize}
  \end{block}{}
}

\frame
{
  \frametitle{Bulk Synchronous Parallelism}
  \begin{block}{Origin}
  \begin{itemize}
  \footnotesize
  \item Proposed by L. Valiant in 1990
  \item Present a constrained form of parallelism
  \item Bridge the gap between machine and programs
 \end{itemize}
  \end{block}{}

  \begin{block}{Principles}
  \begin{itemize}
  \footnotesize
  \item A Machine Model
  \item A Cost Model
  \item A Programming Model
 \end{itemize}
  \end{block}{}
}

\frame
{
  \frametitle{BSP Machine Model}
  \begin{center}\pgfuseimage{bsp_machine}\end{center}

  \begin{block}{Definition}
  \begin{itemize}
  \footnotesize
  \item Multiple Computing units : local memory + processor
  \item One all-to-all interconnection network
  \item A global barrier mechanism
 \end{itemize}
  \end{block}{}
}

\frame
{
  \frametitle{BSP Programming Model}
  \begin{center}\pgfuseimage{bsp_program}\end{center}

  \begin{block}{Definition of a Super-Step}
  \begin{itemize}
  \footnotesize
  \item An asynchronous computation step
  \item A all-to-all communication step
  \item A global barrier
 \end{itemize}
  \end{block}{}
}



\frame
{
  \frametitle{BSP Cost Model}
  \begin{block}{Definition}
  \begin{itemize}
  \footnotesize
  \item $W_i$ : computation time on processor $i$
  \item $h$ : amount of bytes to transfer
  \item $g$ : network throughput
  \item $L$ : Time for performing a barrier
 \end{itemize}
  \end{block}{}

  \begin{block}{Cost of one super-step}
  \begin{center}
  $\Omega \, = \, \max W_i + h \times g + L$
  \end{center}
  \end{block}{}
}

\frame
{
  \frametitle{Existing BSP Library}
  \begin{block}{Oxford BSPLib [Hill:96]}
  \begin{itemize}
  \footnotesize
  \item C based
  \item Rely on low-level shared memory runtime
  \item Provides 20+ primitives for communications over different medium
 \end{itemize}
  \end{block}{}

  \begin{block}{BSML [Gava:09]}
  \begin{itemize}
  \footnotesize
  \footnotesize
  \item Functionnal implementation of BSP in Caml
  \item Notion of parallel 'vector'
  \item Two communications + one synchronization primitives 
  \item Provides an extended syntax for BSP construct in ML
  \end{itemize}
  \end{block}{}
}

\frame
{
  \frametitle{Why BSP ?}
  \begin{block}{BSP Pros and Cons}
  \begin{itemize}
  \footnotesize
  \item Straightforward \texttt{Seq of Par} programming model
  \item Hybrid programming support with a black-box approach
  \item Limited support for task parallelism
  \item Barrier costs impact programm structure
 \end{itemize}
  \end{block}{}

  \begin{block}{Our Plans}
  \begin{itemize}
  \footnotesize
  \item Provide a BSP like library for parallel programming
  \item Provide a tool for BSP application description
  \item Use BSP cost Model to explore configuration space
 \end{itemize}
  \end{block}{}
}
